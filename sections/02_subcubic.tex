\section{APSP Algorithms with Subcubic Runtime}

\begin{frame}{Subcubic Runtimes\footnote[1]{\cite[Section~1]{Chan2007}}}
    \begin{figure}
        \includegraphics[keepaspectratio, width=.55\linewidth]{imgs/runtime_comparison}
    \end{figure}
\end{frame}

\begin{frame}{Subcubic Runtimes}
    \begin{columns}
        \begin{column}{.252\linewidth}
            {
                \setbeamercolor{block title alerted}{fg=matplotlib1}
                \begin{alertblock}{Floyd-Warshall (1962)}
                    $\begin{aligned}
                        \mathcal{O}\left( n^3 \right)
                    \end{aligned}$
                \end{alertblock}
            }

            {
                \setbeamercolor{block title alerted}{fg=matplotlib2}
                \begin{alertblock}{Fredman (1976)}
                    $\begin{aligned}
                        \mathcal{O}\left( n^3 {\left[ \frac{\log(\log(n))}{\log(n)} \right]}^{\frac{1}{3}} \right)
                    \end{aligned}$
                \end{alertblock}
            }
            
            {
                \setbeamercolor{block title alerted}{fg=matplotlib3}
                \begin{alertblock}{Takaoka (1992)}
                    $\begin{aligned}
                        \mathcal{O}\left( n^3 \sqrt{\frac{\log(\log(n))}{\log(n)}} \right)
                    \end{aligned}$
                \end{alertblock}
            }
            
            {
                \setbeamercolor{block title alerted}{fg=matplotlib4}
                \begin{alertblock}{Dobosiewizc (1990)}
                    $\begin{aligned}
                        \mathcal{O}\left( \frac{n^3}{\sqrt{\log(n)}} \right)
                    \end{aligned}$
                \end{alertblock}
            }
        \end{column}
        \begin{column}{.48\linewidth}
            \begin{figure}
                \includegraphics[keepaspectratio, width=.95\linewidth]{imgs/runtime_comparison}
            \end{figure}
        \end{column}
        \begin{column}{.25\linewidth}
            {
                \setbeamercolor{block title alerted}{fg=matplotlib5}
                \begin{alertblock}{Han (2004)}
                    $\begin{aligned}
                        \mathcal{O}\left( n^3 {\left[ \frac{\log(\log(n))}{\log(n)} \right]}^{\frac{5}{7}} \right)
                    \end{aligned}$
                \end{alertblock}
            }
            
            {
                \setbeamercolor{block title alerted}{fg=matplotlib6}
                \begin{alertblock}{Takaoka (2005)}
                    $\begin{aligned}
                        \mathcal{O}\left( n^3 \frac{{\log(\log(n))}^2}{\log(n)} \right)
                    \end{aligned}$
                \end{alertblock}
            }
            
            {
                \setbeamercolor{block title alerted}{fg=matplotlib7}
                \begin{alertblock}{Zwick (2004)}
                    $\begin{aligned}
                        \mathcal{O}\left( n^3 \frac{\sqrt{\log(\log(n))}}{\log(n)} \right)
                    \end{aligned}$
                \end{alertblock}
            }
            
            {
                \setbeamercolor{block title alerted}{fg=matplotlib8}
                \begin{alertblock}{Chan (2007)}
                    $\begin{aligned}
                        \mathcal{O}\left( \frac{n^3}{\log(n)} \right)
                    \end{aligned}$
                \end{alertblock}
            }
        \end{column}
    \end{columns}
\end{frame}
% TODO: consider whether or not it's worth it to actually go into depth about these approaches...
