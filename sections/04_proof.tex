\section{Proofs}

\begin{frame}{Proof of Lemma~\ref{lem:dom_pairs}\footnote[1]{\cite[Lemma~2.1]{Chan2007}}}
    \only<1,2>{
        \setcounter{theorem}{0}
        \begin{lemma}
            Let $\mathbb{P} \subset \mathbb{R}^d$ be a set of red or blue points, partitioned into $\mathbb{P}_{red}$ and $\mathbb{P}_{blue}$ such that $n = \abs{\mathbb{P}_{red}} + \abs{\mathbb{P}_{blue}}$, and $\varepsilon \in \interval[open]{0}{1}$.
            Then we can find all $k$ dominating pairs in a time of $\mathcal{O}\left( c_\varepsilon^d n^{1 + \varepsilon} + k \right)$, where we define $c_\varepsilon := \frac{2^\varepsilon}{2^\varepsilon - 1}$.
        \end{lemma}

        \uncover<2->{
            \begin{exampleblock}{Assumption}
                All values are presorted by the $d$-th coordinate.
            \end{exampleblock}
        }
    }
    \only<3-10>{
        \emph{Idea: Divide and Conquer!}
        
        \uncover<4->{
            \only<4>{
                We split each set $\mathbb{P}_{red}$, $\mathbb{P}_{blue}$ along the median $d$-th coordinate into two sets each:
                \[
                    \mathbb{P}_{red} \mapsto \mathbb{P}_{red, left}, \mathbb{P}_{red, right} \qquad \mathbb{P}_{blue} \mapsto \mathbb{P}_{blue, left}, \mathbb{P}_{blue, right}
                \]
                such that it holds for all $p \in \mathbb{P}_{red}, q \in \mathbb{P}_{blue}$:
                \[
                    p \in
                    \begin{cases}
                        \mathbb{P}_{red, left}, & p_d \leq m_d \\
                        \mathbb{P}_{red, right}, & else
                    \end{cases}
                    \qquad
                    q \in
                    \begin{cases}
                        \mathbb{P}_{blue, left}, & q_d \leq m_d \\
                        \mathbb{P}_{blue, right}, & else
                    \end{cases}
                \]
                
                Here, $m_d$ denote the median $d$-th coordinate for $\mathbb{P}_{red} \cup \mathbb{P}_{blue}$.
            }
            \only<5-7>{
                \begin{figure}
                    \begin{tikzpicture}[scale=0.6, auto, swap]
                        \only<5>{
                            \foreach \pos / \name in {%
                                {(1, 1)/a}, {(4.5, 3.5)/b}%
                            }%
                                \node[redpoint] (\name) at \pos {};
        
                            \foreach \pos / \name in {%
                                {(3, 0.5)/A}, {(2, 3)/B}, {(5, 5)/C}%
                            }%
                                \node[bluepoint] (\name) at \pos {};
                        }

                        \only<6>{
                            \foreach \pos / \name in {%
                                {(1, 1)/a}, {(4.5, 3.5)/b}%
                            }%
                                \node[redpoint] (\name) at \pos {};

                            \node[barlabel] (labelleft) at (-4, 5.5) {$\mathbb{P}_{red, left}$};
                            \node[barlabel] (labelright) at (10, 5.5) {$\mathbb{P}_{red, right}$};
                        }
    
                        \only<7>{
                            \foreach \pos / \name in {%
                                {(3, 0.5)/A}, {(2, 3)/B}, {(5, 5)/C}%
                            }%
                                \node[bluepoint] (\name) at \pos {};

                            \node[barlabel] (labelleft) at (-4, 5.5) {$\mathbb{P}_{blue, left}$};
                            \node[barlabel] (labelright) at (10, 5.5) {$\mathbb{P}_{blue, right}$};
                        }

                        \only<6->{
                            \node[ghost] (hook1) at (3, -.5) {};
                            \node[ghost] (hook2) at (3, 6.5) {};

                            \path[divides] (hook1) -- (hook2);

                            \node[ghost] (lefttarget) at (.5, 5.5) {};
                            \node[ghost] (righttarget) at (5.5, 5.5) {};

                            \path[points] (labelleft) -- (lefttarget);
                            \path[points] (labelright) -- (righttarget);
                        }
    
                        \node[barlabel] (labelx) at (3, -0.75) {Coordinate 2};
                        \node[barlabel, rotate=90] (labely) at (-0.75, 3) {Coordinate 1};
    
                        \node[ghost] (bound1) at (0, 0) {};
                        \node[ghost] (bound2) at (6, 6) {};
    
                        % border
                        \node[draw,fit=(bound1) (bound2)] (border) {};
                    \end{tikzpicture}
                \end{figure}
            }

            \only<8->{
                We solve the dominating pairs problem on the sets
                \[
                    \mathbb{P}_{red, left} \cup \mathbb{P}_{blue, left}, \qquad \mathbb{P}_{red, right} \cup \mathbb{P}_{blue, right}, \qquad \mathbb{P}_{red, left} \cup \mathbb{P}_{blue, right}.
                \]
                (For the first $d - 1$ coordinates for the third set.)

                \uncover<9->{
                    We do not consider $\mathbb{P}_{red, right} \cup \mathbb{P}_{blue, left}$, because $\forall p \in \mathbb{P}_{red, right}, q \in \mathbb{P}_{blue, left}: q_d \leq m_d \leq p_d$.
                    Hence no $q$ can never dominate any $p$.
                }

                \uncover<10->{
                    We stop dividing if the number of points left to consider is $1$; we output all pairs of red and blue points if the number of dimensions left is $0$.
                }
            }
        }
    }
    \only<11-15>{
        Without the output costs of $\mathcal{O}\left( k \right)$, we get the recurrence relation
        \[
            T_d(n) \leq \underbrace{2 T_d\left( \frac{n}{2} \right)}_{(A)} + \underbrace{T_{d - 1}(n)}_{(B)} + \underbrace{\mathcal{O}(n)}_{(C)}.
        \]

        \only<11-13>{
            \uncover<12->{
                $(A)$: Two subproblems for half of the data each. $\rightsquigarrow \mathbb{P}_{red, left} \cup \mathbb{P}_{blue, left}, \mathbb{P}_{red, right} \cup \mathbb{P}_{blue, right}$

                $(B)$: Subproblem where d-th coordinate does not matter. $\rightsquigarrow \mathbb{P}_{red, left} \cup \mathbb{P}_{blue, right}$

                $(C)$: Time to split current point set about the median.\only<12->{\footnote[3]{\cite[Section~2.3.2]{Preparata1985}}}
            }
        
            \uncover<13>{
                From the termination criteria we get
                \[
                    T_d(1) = \mathcal{O}\left( 1 \right), \qquad T_0(n) = \mathcal{O}\left( n \right).
                \]
            }
        }

        \only<14->{
            Let $b$ be fixed, and define $T'(N) := \max \left\{ T_k(i) \,|\, i = 1, \dots, n; k = 1, \dots, d: b^k i \leq N \right\}$.
            
            Substituting into the previous reccurence formula thus yields for some constant $c$:
            \[
                T'(N) \leq 2 T'\left( \frac{N}{2} \right) + T'\left( \frac{N}{b} \right) + cN.
            \]
            
            \uncover<15>{
                $\implies$ What does this reccurence evaluate to and how do we need to choose $b$?
            }
        }
    }
    \only<16-19>{
        \[
            T'(N) \leq 2 T'\left( \frac{N}{2} \right) + T'\left( \frac{N}{b} \right) + cN
        \]

        \uncover<17->{
            By guessing that $T'(N) = \mathcal{O}\left( N^{1 + \varepsilon} - N \right) = \mathcal{O}\left( N^{1 + \varepsilon} \right)$, we get
            \begin{align*}
                T'(N) &\leq 2 \tilde{c} \left( {\left[ \frac{N}{2} \right]}^{1 + \varepsilon} - \frac{N}{2} \right) + \tilde{c} \left( {\left[ \frac{N}{b} \right]}^{1 + \varepsilon} - \frac{N}{b} \right) + cN \\
                \uncover<18->{
                     &= \tilde{c} \left( \frac{2}{2^{1 + \varepsilon}} N^{1 + \varepsilon} - N \right) + \tilde{c} \left( \frac{1}{b^{1 + \varepsilon}} N^{1 + \varepsilon} - \frac{1}{b} N \right) + cN \\
                     \uncover<19->{
                         &\overset{(A)}{=} \tilde{c} N^{1 + \varepsilon} - \tilde{c} N - \underbrace{\frac{\tilde{c}}{b} N}_{\leq \tilde{c} N} + cN \overset{(B)}{=} \tilde{c} \left( N^{1 + \varepsilon} - N \right),
                     }
                }
            \end{align*}
        }
    }
    \only<20->{
        If $(A)$: $\begin{aligned}1 = \frac{2}{2^{1 + \varepsilon}} + \frac{1}{b^{1 + \varepsilon}}\end{aligned}$, and $(B)$: $c \leq \tilde{c}$ are fulfilled:
        \only<20-21>{
            \[
                T'(N) \overset{(A)}{\leq} \tilde{c} N^{1 + \varepsilon} - \tilde{c} N - \underbrace{\frac{\tilde{c}}{b} N}_{\leq \tilde{c} N} + cN \overset{(B)}{=} \tilde{c} \left( N^{1 + \varepsilon} - N \right),
            \]
        }
        \only<21>{
            \begin{align*}
                1 = \frac{2}{2^{1 + \varepsilon}} + \frac{1}{b^{1 + \varepsilon}} &\iff b^{1 + \varepsilon} = b^{1 + \varepsilon} \frac{1}{2^\varepsilon} + 1 \iff -1 = \frac{b^{1 + \varepsilon}}{2^\varepsilon} - b^{1 + \varepsilon} \\
                &\iff -1 = \frac{1 - 2^\varepsilon}{2^\varepsilon} b^{1 + \varepsilon} \iff b^{1 + \varepsilon} = \frac{2^\varepsilon}{2^\varepsilon - 1} = c_\varepsilon
            \end{align*}
        }
        \only<22->{
            \begin{align*}
                T'(N) &= \mathcal{O}\left( N^{1 + \varepsilon} - N \right) = \mathcal{O}\left( N^{1 + \varepsilon} \right) \\
                \implies T_d(n) &= \mathcal{O}\left( {\left( b^d n \right)}^{1 + \varepsilon} \right) = \mathcal{O}\left( c_\varepsilon^d n^{1 + \varepsilon} \right),
            \end{align*}
            because $T'(N) := \max \left\{ T_k(i) \,|\, i = 1, \dots, n; k = 1, \dots, d: b^k i \leq N \right\}$. 
            
            Finally, outputing all $k$ dominating pairs requires $\mathcal{O}\left( k \right)$ time. \qed{}
        }
    }
\end{frame}

\begin{frame}{Proof of Lemma~\ref{lem:sub_mat_mul}\footnote[1]{\cite[Lemma~3.1]{Chan2007}}}
    \only<1, 2>{
        \setcounter{theorem}{1}
        \begin{lemma}
            Given two matrices $A \in \mathbb{R}^{n \times d}, B \in \mathbb{R}^{d \times n}$, and $\varepsilon \in \interval[open]{0}{1}, \begin{aligned}c_\varepsilon := \frac{2^\varepsilon}{2^\varepsilon - 1}\end{aligned}$, we can compute their min-plus product $A \otimes B$ in $\mathcal{O}\left( d c_\varepsilon^d n^{1 + \varepsilon} + n^2 \right)$ time.
        \end{lemma}
        
        \uncover<2>{
            \begin{exampleblock}{Note}
                Ties such as $a_{i, k} + b_{k, j} = a_{i, \tilde{k}} + b_{\tilde{k}, j}$ are considered to be ``$<$'' if $k \leq \tilde{k}$.
                In the paper this is written as $k < \tilde{k}$ under the assumption that $\tilde{k} = 1, \dots, k - 1, k + 1, \dots, d$.
            \end{exampleblock}
        }
    }
    \only<3-5>{
        Consider $A$ to have entries ${\left( a_{i, j} \right)}_{i, j = 1}^{n, d}$, and $B$ to have entries ${\left( a_{i, j} \right)}_{i, j = 1}^{d, n}$.
        We want to compute the pairs
        \begin{align*}
            X_k &= \{ (i, j) \,|\, \forall k' = 1, \dots, d: a_{i, k} + b_{k, j} \leq a_{i, k'} + b_{k', j} \} \\
             &= \{ (i, j) \,|\, \forall k' = 1, \dots, d: a_{i, k} - a_{i, k'} \leq b_{k', j} - b_{k, j} \}.
        \end{align*}
        \uncover<4->{
            This effectively amounts to computing all dominant pairs between the sets
            \begin{align*}
                \mathcal{A}_k &:= {\left\{ (a_{i, k} - a_{i, 1}), (a_{i, k} - a_{i, 2}), \dots, (a_{i, k} - a_{i, d}) \right\}}_{i = 1}^{n}, \text{and} \\
                \mathcal{B}_k &:= {\left\{ (b_{1, j} - b_{k, j}), (b_{2, j} - b_{k, j}), \dots, (b_{d, j} - b_{k, j}) \right\}}_{j = 1}^{n},
            \end{align*}
            where $\mathcal{A}_k$ takes the role of red points and $\mathcal{B}_k$ acts as the set of blue points.
        }

        \uncover<5>{
            By Lemma~\ref{lem:dom_pairs}, this takes an effort of $\mathcal{O}\left( c_\varepsilon^d n^{1 + \varepsilon} + \abs{X_k} \right)$.
        }
    }
    \only<6>{
        The penultimate step is to compute that
        \begin{align*}
            \mathcal{O}\left( \sum\limits_{k = 1}^d \left( c_\varepsilon^d n^{1 + \varepsilon} + \abs{X_k} \right) \right) = \mathcal{O}\left( d c_\varepsilon^d n^{1 + \varepsilon} + \sum\limits_{k = 1}^d \abs{X_k} \right),
        \end{align*}
        leaving only to compute that $\begin{aligned}\sum\limits_{k = 1}^d \abs{X_k}\end{aligned} = n^2$.
    }
    \only<7->{
        Suppose an index pair $(i, j)$ were to be included in $X_k$, and in $X_{\tilde{k}}$, with $k \neq \tilde{k}$.
        Thus we especially get that $a_{i, k} + b_{k, j} \leq a_{i, \tilde{k}} + b_{\tilde{k}, j}$.
        Applying the definition of $X_{\tilde{k}}$, we also get $a_{i, \tilde{k}} + b_{\tilde{k}, j} \leq a_{i, k} + b_{k, j}$.
        This means that $a_{i, \tilde{k}} + b_{\tilde{k}, j} = a_{i, k} + b_{k, j}$.
        To break this tie, we can w.l.o.g. assume $k < \tilde{k}$, which would result in $(i, j) \not\in X_{\tilde{k}}$.

        \uncover<8->{
            Now assume that there would exist an index pair $(i, j)$ that is contained in none of the $X_k$.
            Then the condition $\forall k = 1, \dots, d: \forall \tilde{k} = 1, \dots, d: a_{i, k} + b_{k, j} > a_{i, \tilde{k}} + b_{\tilde{k}, j}$.
            But obviously for $k = \tilde{k}$ we have ``$=$''.
        }

        \uncover<9>{
            This shows that aggregating over all $X_k$ counts every index pair exactly once.
            $\implies \begin{aligned}\sum\limits_{k = 1}^d \abs{X_k}\end{aligned} = n^2 \implies \mathcal{O}\left( \sum\limits_{k = 1}^d \left( c_\varepsilon^d n^{1 + \varepsilon} + \abs{X_k} \right) \right) = \mathcal{O}\left( d c_\varepsilon^d n^{1 + \varepsilon} + n^2 \right)$. \qed{}
        }
    }
\end{frame}

\begin{frame}{Proof of Theorem~\ref{thm:mat_mul}\footnote[1]{\cite[Theorem~3.2]{Chan2007}}}
    \only<1>{
        \setcounter{theorem}{2}
        \begin{theorem}
            Given any two matrices $A, B \in \mathbb{R}^{n \times n}$ we can compute their min-plus (distance) product in a time of $\mathcal{O}\left( n^3 / \log(n) \right)$.
        \end{theorem}
    }
    \only<2>{
        We recall the idea of splitting matrices to multiply them:

        \begin{columns}
            \begin{column}{.2\linewidth}
                \begin{figure}
                    \begin{tikzpicture}
                        \matrix[hsupermatrix]{
                            \node[vsubmatrix] (a1) {A_1}; \& \node[vsubmatrix] (a2) {A_2}; \& \cdots \& \node[vsubmatrix] (ad) {A_d}; \\
                        };
                    \end{tikzpicture}
                \end{figure}
            \end{column}
            \begin{column}{.2\linewidth}
                \begin{figure}
                    \begin{tikzpicture}
                        \matrix[vsupermatrix]{
                            \node[hsubmatrix] (b1) {B_1}; \\ \node[hsubmatrix] (b2) {B_2}; \\ \vdots \\ \node[hsubmatrix] (bd) {B_d}; \\
                        };
                    \end{tikzpicture}
                \end{figure}
            \end{column}
        \end{columns}

        $\implies$ Strassen; relation between matrix multiplication and matrix closure.
    }
    \only<3-6>{
        We split our matrices $A$ and $B$ into $\begin{aligned}\frac{n}{d}\end{aligned}$ blocks, that is $\begin{aligned}\forall i = 1, \dots, \frac{n}{d}: A_i \in \mathbb{R}^{n \times \frac{n}{d}}, B_i \in \mathbb{R}^{\frac{n}{d} \times n}\end{aligned}$.
        (If necessary we round $\begin{aligned}\frac{n}{d}\end{aligned}$ to the closest integer and adjust the number of blocks accordingly.)

        \uncover<4->{
            We then compute the distance products $A_i \otimes B_i$ for all $\begin{aligned}i = 1, \dots, \frac{n}{d}\end{aligned}$, and set the product to be defined by the element-wise minimum, i.e.\ $\begin{aligned}[t]c_{i, j} := \min\limits_{l = 1, \dots, \frac{n}{d}} {\left( A_l \otimes B_l \right)}_{i, j}\end{aligned}$, where $i, j = 1, \dots, n$.
        }

        \uncover<5->{
            By Lemma~\ref{lem:sub_mat_mul}, this procedure requires $\begin{aligned}\mathcal{O}\left( \frac{n}{d} \left( d c_\varepsilon^d n^{1 + \varepsilon} + n^2 \right) \right) = \mathcal{O} \left( c_\varepsilon^d n^{2 + \varepsilon} + \frac{n^3}{d} \right)\end{aligned}$ time.
        }

        \uncover<6->{
            It now only remains to choose the constant $d$.
        }
    }
    \only<7->{
        We want to assure $\begin{aligned}\frac{n^3}{d} > c_\varepsilon^d n^{2 + \varepsilon}\end{aligned}$.

        \uncover<7->{
            The choice of $d = \tilde{c} \log(n)$ is useful, where $\tilde{c}$ is sufficiently small and depending on $\varepsilon$, because we then get
            \[
                \mathcal{O}\left( c_\varepsilon^d n^{2 + \varepsilon} + \frac{n^3}{d} \right) = \mathcal{O}\left( \frac{n^3}{\log(n)} \right).
            \]

            One possible choice is $\varepsilon \approx 0.38, c_\varepsilon \approx 4.32, d \approx 0.42 \log(n)$\footnotemark[1]. \qed{}
        }
    }
\end{frame}

\begin{frame}{Proof of Corollary~\ref{cor:apsp_subcubic}\footnote[1]{\cite[Corollary~3.3]{Chan2007}}}
    \only<1>{
        \setcounter{theorem}{3}
        \begin{corollary}
            We can solve the all pairs shortest paths problem for a graph $G = (V, E)$ with $\abs{V} = n$ nodes in $\mathcal{O}\left( n^3 / \log(n) \right)$ time.
        \end{corollary}
    }
    \only<2->{
        We consider $A$ and $B$ to be the matrices defined by $\begin{aligned}d(i, j) := \begin{cases}
            w_{i, j}, &\exists e \in E: e = (i, j) \\
            \infty, &else
        \end{cases}\end{aligned}$.

        The corollary then follows by applying Theorem~\ref{thm:mat_mul}. \qed{}
    }
\end{frame}