\section{Proof}

\begin{frame}{Proof of Lemma~\ref{lem:dom_pairs}\footnote[1]{\cite[Lemma~2.1]{Chan2007}}}
    \only<1,2>{
        \setcounter{theorem}{0}
        \begin{lemma}
            Let $\mathbb{P} \subset \mathbb{R}^d$ be a set of red or blue points, partitioned into $\mathbb{P}_{red}$ and $\mathbb{P}_{blue}$ such that $n = \abs{\mathbb{P}_{red}} + \abs{\mathbb{P}_{blue}}$, and $\varepsilon \in \interval[open]{0}{1}$.
            Then we can find all $k$ dominating pairs in a time of $\mathcal{O}\left( c_\varepsilon^d n^{1 + \varepsilon} + k \right)$, where we define $c_\varepsilon := \frac{2^\varepsilon}{2^\varepsilon - 1}$.
        \end{lemma}

        \uncover<2->{
            \begin{exampleblock}{Assumption}
                All palues are presorted by the $d$-th coordinate using a linear-time median-finding algorithm.
            \end{exampleblock}
        }
    }

    \only<3-10>{
        \emph{Idea: Divide and Conquer!}
        
        \uncover<4->{
            \only<4>{
                We split each set $\mathbb{P}_{red}$, $\mathbb{P}_{blue}$ along the median $d$-th coordinate into two sets each:
                \[
                    \mathbb{P}_{red} \mapsto \mathbb{P}_{red, left}, \mathbb{P}_{red, right} \qquad \mathbb{P}_{blue} \mapsto \mathbb{P}_{blue, left}, \mathbb{P}_{blue, right}
                \]
                such that it holds for all $p \in \mathbb{P}_{red}, q \in \mathbb{P}_{blue}$:
                \[
                    p \in
                    \begin{cases}
                        \mathbb{P}_{red, left}, & p_d \leq m_d \\
                        \mathbb{P}_{red, right}, & else
                    \end{cases}
                    \qquad
                    q \in
                    \begin{cases}
                        \mathbb{P}_{blue, left}, & q_d \leq m_d \\
                        \mathbb{P}_{blue, right}, & else
                    \end{cases}
                \]
                
                Here, $m_d$ denote the median $d$-th coordinate for $\mathbb{P}_{red} \cup \mathbb{P}_{blue}$.
            }
            
            \only<5-7>{
                \begin{figure}
                    \begin{tikzpicture}[scale=0.6, auto, swap]
                        \only<5>{
                            \foreach \pos / \name in {%
                                {(1, 1)/a}, {(4.5, 3.5)/b}%
                            }%
                                \node[redpoint] (\name) at \pos {};
        
                            \foreach \pos / \name in {%
                                {(3, 0.5)/A}, {(2, 3)/B}, {(5, 5)/C}%
                            }%
                                \node[bluepoint] (\name) at \pos {};
                        }

                        \only<6>{
                            \foreach \pos / \name in {%
                                {(1, 1)/a}, {(4.5, 3.5)/b}%
                            }%
                                \node[redpoint] (\name) at \pos {};

                            \node[barlabel] (labelleft) at (-4, 5.5) {$\mathbb{P}_{red, left}$};
                            \node[barlabel] (labelright) at (10, 5.5) {$\mathbb{P}_{red, right}$};
                        }
    
                        \only<7>{
                            \foreach \pos / \name in {%
                                {(3, 0.5)/A}, {(2, 3)/B}, {(5, 5)/C}%
                            }%
                                \node[bluepoint] (\name) at \pos {};

                            \node[barlabel] (labelleft) at (-4, 5.5) {$\mathbb{P}_{blue, left}$};
                            \node[barlabel] (labelright) at (10, 5.5) {$\mathbb{P}_{blue, right}$};
                        }

                        \only<6, 7>{
                            \node[ghost] (hook1) at (3, -.5) {};
                            \node[ghost] (hook2) at (3, 6.5) {};

                            \path[divides] (hook1) -- (hook2);

                            \node[ghost] (lefttarget) at (.5, 5.5) {};
                            \node[ghost] (righttarget) at (5.5, 5.5) {};

                            \path[points] (labelleft) -- (lefttarget);
                            \path[points] (labelright) -- (righttarget);
                        }
    
                        \node[barlabel] (labelx) at (3, -0.75) {Coordinate 2};
                        \node[barlabel, rotate=90] (labely) at (-0.75, 3) {Coordinate 1};
    
                        \node[ghost] (bound1) at (0, 0) {};
                        \node[ghost] (bound2) at (6, 6) {};
    
                        % border
                        \node[draw,fit=(bound1) (bound2)] (border) {};
                    \end{tikzpicture}
                \end{figure}
            }
            
            \only<8-10>{
                We solve the dominating pairs problem on the sets
                \[
                    \mathbb{P}_{red, left} \cup \mathbb{P}_{blue, left}, \qquad \mathbb{P}_{red, right} \cup \mathbb{P}_{blue, right}, \qquad \mathbb{P}_{red, left} \cup \mathbb{P}_{blue, right}.
                \]
                (For the first $d - 1$ coordinates for the third set.)

                \uncover<9, 10>{
                    We do not consider $\mathbb{P}_{red, right} \cup \mathbb{P}_{blue, left}$, because $\forall p \in \mathbb{P}_{red, right}, q \in \mathbb{P}_{blue, left}: q_d \leq m_d \leq p_d$.
                    Hence no $q$ can never dominate any $p$.
                }

                \uncover<10>{
                    We stop dividing if the number of points left to consider is $1$; we output all pairs of red and blue points if the number of dimensions left is $0$.
                }
            }

            % TODO: think about if we need more figures here, that explain the termination conditions...
        }
    }

    \only<11->{
        Without the output costs of $\mathcal{O}\left( k \right)$, we get the recurrence relation
        \[
            T_d(n) \leq 2 T_d\left( \frac{n}{2} \right) + T_{d - 1}(n) + \mathcal{O}(n).
        \]
    }
\end{frame}