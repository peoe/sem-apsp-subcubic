\section{Computing the Runtime}

\begin{frame}{Proof of Lemma~\ref{lem:dom_pairs}\footnote[1]{\cite[Lemma~2.1]{Chan2007}}}
    \setcounter{theorem}{0}
    \begin{lemma}
        Let $\mathbb{P} \subset \mathbb{R}^d$ be a set of red or blue points, partitioned into $\mathbb{P}_{red}$ and $\mathbb{P}_{blue}$ such that $n = \abs{\mathbb{P}_{red}} + \abs{\mathbb{P}_{blue}}$, and $\varepsilon \in \interval[open]{0}{1}$.
        Then we can find all $k$ dominating pairs in a time of $\mathcal{O}\left( c_\varepsilon^d n^{1 + \varepsilon} + k \right)$, where we define $c_\varepsilon := \frac{2^\varepsilon}{2^\varepsilon - 1}$.
    \end{lemma}
\end{frame}

\begin{frame}{Proof of Lemma~\ref{lem:dom_pairs}}
    \begin{exampleblock}{Assumption}
        All values are presorted by the $d$-th coordinate.
    \end{exampleblock}
\end{frame}

\begin{frame}{Proof of Lemma~\ref{lem:dom_pairs}}
    \emph{Idea: Divide and Conquer!}
    
    \uncover<1->{
        We split each set $\mathbb{P}_{red}$, $\mathbb{P}_{blue}$ along the median $d$-th coordinate into two sets each:
        \[
            \mathbb{P}_{red} \mapsto \mathbb{P}_{red, left}, \mathbb{P}_{red, right} \qquad \mathbb{P}_{blue} \mapsto \mathbb{P}_{blue, left}, \mathbb{P}_{blue, right}
        \]
    }
    \uncover<2->{
        such that it holds for all $p \in \mathbb{P}_{red}, q \in \mathbb{P}_{blue}$:
        \[
            p \in
            \begin{cases}
                \mathbb{P}_{red, left}, & p_d \leq m_d \\
                \mathbb{P}_{red, right}, & else
            \end{cases}
            \qquad
            q \in
            \begin{cases}
                \mathbb{P}_{blue, left}, & q_d \leq m_d \\
                \mathbb{P}_{blue, right}, & else
            \end{cases}
        \]

        Here, $m_d$ denote the median $d$-th coordinate for $\mathbb{P}_{red} \cup \mathbb{P}_{blue}$.
    }
\end{frame}

\begin{frame}{Proof of Lemma~\ref{lem:dom_pairs}}
    \emph{Idea: Divide and Conquer!}
    
    \begin{figure}
        \begin{tikzpicture}[scale=0.6, auto, swap]
            \only<1, 2, 3>{
                \foreach \pos / \name in {%
                    {(1, 1)/a}, {(4.5, 3.5)/b}%
                }%
                    \node[redpoint] (\name) at \pos {};
            }

            \only<1, 2, 4>{
                \foreach \pos / \name in {%
                    {(3, 0.5)/A}, {(2, 3)/B}, {(5, 5)/C}%
                }%
                    \node[bluepoint] (\name) at \pos {};
            }

            \only<3>{
                \foreach \pos / \name in {%
                    {(3, 0.5)/A}, {(2, 3)/B}, {(5, 5)/C}%
                }%
                    \node[bluepointalpha] (\name) at \pos {};

                \node[barlabel] (labelleft) at (-4, 5.5) {$\mathbb{P}_{red, left}$};
                \node[barlabel] (labelright) at (10, 5.5) {$\mathbb{P}_{red, right}$};
            }

            \only<4>{
                \foreach \pos / \name in {%
                    {(1, 1)/a}, {(4.5, 3.5)/b}%
                }%
                    \node[redpointalpha] (\name) at \pos {};

                \node[barlabel] (labelleft) at (-4, 5.5) {$\mathbb{P}_{blue, left}$};
                \node[barlabel] (labelright) at (10, 5.5) {$\mathbb{P}_{blue, right}$};
            }

            \only<2->{
                \node[ghost] (hook1) at (3, -.5) {};
                \node[ghost] (hook2) at (3, 6.5) {};

                \path[divides] (hook1) -- (hook2);
            }
            
            \only<3->{
                \node[ghost] (lefttarget) at (.5, 5.5) {};
                \node[ghost] (righttarget) at (5.5, 5.5) {};

                \path[points] (labelleft) -- (lefttarget);
                \path[points] (labelright) -- (righttarget);
            }

            \node[barlabel] (labelx) at (3, -0.75) {Coordinate 2};
            \node[barlabel, rotate=90] (labely) at (-0.75, 3) {Coordinate 1};

            \node[ghost] (bound1) at (0, 0) {};
            \node[ghost] (bound2) at (6, 6) {};

            % border
            \node[draw, fit=(bound1) (bound2)] (border) {};
        \end{tikzpicture}
    \end{figure}
\end{frame}

\begin{frame}{Proof of Lemma~\ref{lem:dom_pairs}}
    We solve the dominating pairs problem on the sets
    \[
        \mathbb{P}_{red, left} \cup \mathbb{P}_{blue, left}, \qquad \mathbb{P}_{red, right} \cup \mathbb{P}_{blue, right}, \qquad \mathbb{P}_{red, left} \cup \mathbb{P}_{blue, right}.
    \]
    (For the first $d - 1$ coordinates for the third set.)

    \uncover<2->{
        We do not consider $\mathbb{P}_{red, right} \cup \mathbb{P}_{blue, left}$, because $\forall p \in \mathbb{P}_{red, right}, q \in \mathbb{P}_{blue, left}: q_d \leq m_d \leq p_d$.
        Hence no $q$ can never dominate any $p$.
    }

    \uncover<3->{
        We stop dividing if the number of points left to consider is $1$; we output all pairs of red and blue points if the number of dimensions left is $0$.
    }
\end{frame}

\begin{frame}{Proof of Lemma~\ref{lem:dom_pairs}}
    \[
        \mathbb{P}_{red, left} \cup \mathbb{P}_{blue, left}, \qquad \mathbb{P}_{red, right} \cup \mathbb{P}_{blue, right}, \qquad \mathbb{P}_{red, left} \cup \mathbb{P}_{blue, right}
    \]

    We do not need to consider $\mathbb{P}_{red, right} \cup \mathbb{P}_{blue, left}$.

    \uncover<2->{
        \begin{figure}
            \centering
            \begin{tikzpicture}[scale=0.6, auto, swap]
                \only<2, 4>{\node[redpoint] (a) at (1, 1) {};}
                \only<3>{\node[redpoint] (b) at (4.5, 3.5) {};}

                \only<3, 4>{\node[bluepoint] (C) at (5, 5) {};}
                \only<2>{
                    \node[bluepoint] (A) at (3, 0.5) {};
                    \node[bluepoint] (B) at (2, 3) {};
                }

                \node[barlabel] (labelx) at (3, -0.75) {Coordinate 2};
                \node[barlabel, rotate=90] (labely) at (-0.75, 3) {Coordinate 1};

                \node[ghost] (bound1) at (0, 0) {};
                \node[ghost] (bound2) at (6, 6) {};

                \node[ghost] (hook1) at (3, -.5) {};
                \node[ghost] (hook2) at (3, 6.5) {};

                \path[divides] (hook1) -- (hook2);

                \node[ghost] (middlehook1) at (2.6, 6) {};
                \node[ghost] (middlehook2) at (3.4, 0) {};

                % alpha highlights
                \only<2, 4>{
                    \node[fill=red, fill opacity=.2, fit=(bound1) (middlehook1)] (redleft) {};
                    \only<2>{\node[fill=blue, fill opacity=.2, fit=(bound1) (middlehook1)] (blueleft) {};}
                }
                \only<3, 4>{
                    \node[fill=blue, fill opacity=.2, fit=(middlehook2) (bound2)] (blueright) {};
                    \only<3>{\node[fill=red, fill opacity=.2, fit=(middlehook2) (bound2)] (redright) {};}
                }

                % border
                \node[draw, fit=(bound1) (bound2)] (border) {};
            \end{tikzpicture}
        \end{figure}
    }
\end{frame}

\begin{frame}{Proof of Lemma~\ref{lem:dom_pairs}}
    Without the output costs of $\mathcal{O}\left( k \right)$, we get the recurrence relation
    \[
        T_d(n) \leq \underbrace{2 T_d\left( \frac{n}{2} \right)}_{(A)} + \underbrace{T_{d - 1}(n)}_{(B)} + \underbrace{\mathcal{O}(n)}_{(C)}.
    \]

    \uncover<2->{
        $(A)$: Two subproblems for half of the data each. $\leadsto \mathbb{P}_{red, left} \cup \mathbb{P}_{blue, left}, \mathbb{P}_{red, right} \cup \mathbb{P}_{blue, right}$

        $(B)$: Subproblem where d-th coordinate does not matter. $\leadsto \mathbb{P}_{red, left} \cup \mathbb{P}_{blue, right}$

        $(C)$: Time to split current point set about the median.\only<2->{\footnote[1]{\cite[Section~2.3.2]{Preparata1985}}}
    }

    \uncover<3>{
        From the termination criteria we get
        \[
            T_d(1) = \mathcal{O}\left( 1 \right), \qquad T_0(n) = \mathcal{O}\left( n \right).
        \]
    }

    % TODO: add images here to explain the degenerated case "all red points are left, all blue points are right"
\end{frame}

\begin{frame}{Proof of Lemma~\ref{lem:dom_pairs}}
    \[
        T_0(n) = \mathcal{O}\left( n \right)?
    \]

    ``If $d = 0$, we just output all pairs of red and blue points.''\only<1>{\footnote[1]{\cite[Lemma~2.1]{Chan2007}}}

    \uncover<2->{
        $\leadsto$ Output: $n^2$ pairs in the worst case \uncover<3->{$\implies T_0(n) = \mathcal{O}\left( n^2 \right)$}
    }

    \uncover<4->{
        $\leadsto$ Output: (\texttt{Pairs}, $\mathbb{P}_{red, left}$, $\mathbb{P}_{blue, right}$) in $\mathcal{O}(n)$\uncover<5->{, but postprocessing necessary later on!}
    }

    \uncover<6>{
        $\implies$ At some point we have to compute upto $\mathcal{O} \left( n^2 \right)$\dots
    }
\end{frame}

\begin{frame}{Proof of Lemma~\ref{lem:dom_pairs}}
    \[
        T_d(n) \leq 2 T_d\left( \frac{n}{2} \right) + T_{d - 1}(n) + \mathcal{O}(n)
    \]

    A first solution to this equation is that $T_d(n) = \mathcal{O}\left( n {\log(n)}^d \right)$, yielding an algorithmic runtime of $\mathcal{O}\left( n {\log(n)}^d + k \right)$. (Additional logarithmic factors can be safed by handling the cases $d = 1$ and $d = 2$ independently.\footnotemark[1]{})

    However, we can do better\dots
\end{frame}

\begin{frame}{Proof of Lemma~\ref{lem:dom_pairs}}
    \[
        T_d(n) \leq 2 T_d\left( \frac{n}{2} \right) + T_{d - 1}(n) + \mathcal{O}(n)
    \]

    Let $b$ be fixed, and define $T'(N) := \max \left\{ T_k(i) \,|\, i = 1, \dots, n; k = 1, \dots, d: b^k i \leq N \right\}$.
    
    Substituting into the previous reccurence formula thus yields for some constant $c$:
    \[
        T'(N) \leq 2 T'\left( \frac{N}{2} \right) + T'\left( \frac{N}{b} \right) + cN.
    \]
    
    \uncover<2>{
        $\implies$ What does this reccurence evaluate to and how do we need to choose $b$?
    }
\end{frame}

\begin{frame}{Proof of Lemma~\ref{lem:dom_pairs}}
    \[
        T'(N) \leq 2 T'\left( \frac{N}{2} \right) + T'\left( \frac{N}{b} \right) + cN
    \]

    \uncover<2->{
        By guessing that $T'(N) = \mathcal{O}\left( N^{1 + \varepsilon} - N \right) = \mathcal{O}\left( N^{1 + \varepsilon} \right)$, we get
        \begin{align*}
            T'(N) &\leq 2 \tilde{c} \left( {\left[ \frac{N}{2} \right]}^{1 + \varepsilon} - \frac{N}{2} \right) + \tilde{c} \left( {\left[ \frac{N}{b} \right]}^{1 + \varepsilon} - \frac{N}{b} \right) + cN \\
            \uncover<3->{
                &= \tilde{c} \left( \frac{2}{2^{1 + \varepsilon}} N^{1 + \varepsilon} - N \right) + \tilde{c} \left( \frac{1}{b^{1 + \varepsilon}} N^{1 + \varepsilon} - \frac{1}{b} N \right) + cN \\
            }
            \uncover<4->{
                &\leq \left( \frac{2}{2^{1 + \varepsilon}} + \frac{1}{b^{1 + \varepsilon}} \right) \tilde{c} N^{1 + \varepsilon} - \tilde{c} N - \frac{\tilde{c}}{b} N + cN.
            }
        \end{align*}
    }
\end{frame}

\begin{frame}{Proof of Lemma~\ref{lem:dom_pairs}}
    Thus far we computed
    \[
        T'(N) \leq \underbrace{\left( \frac{2}{2^{1 + \varepsilon}} + \frac{1}{b^{1 + \varepsilon}} \right)}_{(A)} \tilde{c} N^{1 + \varepsilon} - \tilde{c} N - \underbrace{\frac{\tilde{c}}{b} N + cN}_{(B)}.
    \]

    \uncover<2->{
        If $(A)$: $\begin{aligned}1 = \frac{2}{2^{1 + \varepsilon}} + \frac{1}{b^{1 + \varepsilon}}\end{aligned}$, and $(B)$: $c \leq \frac{\tilde{c}}{b}$ are fulfilled we get
        \[
            T'(N) \leq \tilde{c} N^{1 + \varepsilon} - \tilde{c} N \uncover<3->{\implies T'(N) = \mathcal{O}\left( N^{1 + \varepsilon} - N \right) = \mathcal{O}\left( N^{1 + \varepsilon} \right).}
        \]
    }

    \uncover<4>{
        It remains to calculate $b$.
    }
\end{frame}

\begin{frame}{Proof of Lemma~\ref{lem:dom_pairs}}
    If $(A)$: $\begin{aligned}1 = \frac{2}{2^{1 + \varepsilon}} + \frac{1}{b^{1 + \varepsilon}}\end{aligned}$, and $(B)$: $c \leq \frac{\tilde{c}}{b}$ are fulfilled we get
    \[
        T'(N) \leq \tilde{c} N^{1 + \varepsilon} - \tilde{c} N \implies T'(N) = \mathcal{O}\left( N^{1 + \varepsilon} - N \right) = \mathcal{O}\left( N^{1 + \varepsilon} \right).
    \]

    In $(A)$ we get $\begin{aligned}[t]
        1 = \frac{2}{2^{1 + \varepsilon}} + \frac{1}{b^{1 + \varepsilon}} &\iff b^{1 + \varepsilon} = b^{1 + \varepsilon} \frac{1}{2^\varepsilon} + 1 \iff -1 = \frac{b^{1 + \varepsilon}}{2^\varepsilon} - b^{1 + \varepsilon} \\
        &\iff -1 = \frac{1 - 2^\varepsilon}{2^\varepsilon} b^{1 + \varepsilon} \iff b^{1 + \varepsilon} = \frac{2^\varepsilon}{2^\varepsilon - 1} =: c_\varepsilon.
    \end{aligned}$
\end{frame}

\begin{frame}{Proof of Lemma~\ref{lem:dom_pairs}}
    We can now resubstitute:    
    \begin{align*}
        T'(N) &= \mathcal{O}\left( N^{1 + \varepsilon} - N \right) = \mathcal{O}\left( N^{1 + \varepsilon} \right) \\
        \implies T_d(n) &= \mathcal{O}\left( {\left( b^d n \right)}^{1 + \varepsilon} \right) = \mathcal{O}\left( c_\varepsilon^d n^{1 + \varepsilon} \right),
    \end{align*}
    because $T'(N) := \max \left\{ T_k(i) \,|\, i = 1, \dots, n; k = 1, \dots, d: b^k i \leq N \right\}$. 
    
    Finally, outputing all $k$ dominating pairs requires $\mathcal{O}\left( k \right)$ time. \qed{}
\end{frame}

\begin{frame}{Proof of Lemma~\ref{lem:sub_mat_mul}\footnote[1]{\cite[Lemma~3.1]{Chan2007}}}
    \setcounter{theorem}{1}
    \begin{lemma}
        Given two matrices $A \in \mathbb{R}^{n \times d}, B \in \mathbb{R}^{d \times n}$, and $\varepsilon \in \interval[open]{0}{1}, \begin{aligned}c_\varepsilon := \frac{2^\varepsilon}{2^\varepsilon - 1}\end{aligned}$, we can compute their min-plus product $A \otimes B$ in $\mathcal{O}\left( d c_\varepsilon^d n^{1 + \varepsilon} + n^2 \right)$ time.
    \end{lemma}
\end{frame}

\begin{frame}{Proof of Lemma~\ref{lem:sub_mat_mul}}
    \begin{exampleblock}{Note}
        We need to evaluate the inequality $a_{i, k} + b_{k, j} \leq a_{i, \tilde{k}} + b_{\tilde{k}, j}$.

        Ties such as $a_{i, k} + b_{k, j} = a_{i, \tilde{k}} + b_{\tilde{k}, j}$ are considered to be ``$<$'' if $k < \tilde{k}$.
    \end{exampleblock}
\end{frame}

\begin{frame}{Proof of Lemma~\ref{lem:sub_mat_mul}}
    \begin{exampleblock}{Reminder}
        We compute $C = A \otimes B$ elementwise through $c_{i, j} = \min\limits_k a_{i, k} + b_{k, j}$.
    \end{exampleblock}

    \uncover<2->{
        Consider $A$ to have entries ${\left( a_{i, j} \right)}_{i, j = 1}^{n, d}$, and $B$ to have entries ${\left( b_{i, j} \right)}_{i, j = 1}^{d, n}$.
        We want to compute the pairs
        \[
            X_k = \{ (i, j) \,|\, \forall k' = 1, \dots, d: a_{i, k} + b_{k, j} \leq a_{i, k'} + b_{k', j} \}
        \]
    }

    \uncover<3->{
        After computing these $X_k$, we can then set $C = A \otimes B$ elementwise as follows:

        For $(i, j) \in X_K$ we set $c_{i, j} = a_{i, k} + b_{k, j}$.
    }
\end{frame}

\begin{frame}{Proof of Lemma~\ref{lem:sub_mat_mul}}
    \begin{align*}
        X_k &= \{ (i, j) \,|\, \forall k' = 1, \dots, d: a_{i, k} + b_{k, j} \leq a_{i, k'} + b_{k', j} \} \\
            \uncover<2->{&= \{ (i, j) \,|\, \forall k' = 1, \dots, d: a_{i, k} - a_{i, k'} \leq b_{k', j} - b_{k, j} \}}
    \end{align*}
    \uncover<3->{
        This effectively amounts to computing all dominant pairs between the sets
        \begin{align*}
            \mathcal{A}_k &:= {\left\{ (a_{i, k} - a_{i, 1}), (a_{i, k} - a_{i, 2}), \dots, (a_{i, k} - a_{i, d}) \right\}}_{i = 1}^{n}, \text{and} \\
            \mathcal{B}_k &:= {\left\{ (b_{1, j} - b_{k, j}), (b_{2, j} - b_{k, j}), \dots, (b_{d, j} - b_{k, j}) \right\}}_{j = 1}^{n},
        \end{align*}
        where $\mathcal{A}_k$ takes the role of red points and $\mathcal{B}_k$ acts as the set of blue points.
    }

    \uncover<4>{
        By Lemma~\ref{lem:dom_pairs}, this takes an effort of $\mathcal{O}\left( c_\varepsilon^d n^{1 + \varepsilon} + \abs{X_k} \right)$.
    }
\end{frame}

\begin{frame}{Proof of Lemma~\ref{lem:sub_mat_mul}}
    \begin{exampleblock}{Note}
        It is possible to recover the shortest paths from the construction of the $X_k$s.
        We will see more on this later.
    \end{exampleblock}

    \uncover<2->{
        The penultimate step is to compute that
        \begin{align*}
            \mathcal{O}\left( \sum\limits_{k = 1}^d \left( c_\varepsilon^d n^{1 + \varepsilon} + \abs{X_k} \right) \right) = \mathcal{O}\left( d c_\varepsilon^d n^{1 + \varepsilon} + \sum\limits_{k = 1}^d \abs{X_k} \right),
        \end{align*}
        leaving only to compute that $\begin{aligned}\sum\limits_{k = 1}^d \abs{X_k}\end{aligned} = n^2$.
    }
\end{frame}

\begin{frame}{Proof of Lemma~\ref{lem:sub_mat_mul}}
    Suppose an index pair $(i, j)$ were to be included in $X_k$, and in $X_{\tilde{k}}$, with $k \neq \tilde{k}$.

    \uncover<2->{
        Recall the definition of $X_k$: 
        \[
            X_k = \{ (i, j) \,|\, \forall k' = 1, \dots, d: a_{i, k} + b_{k, j} \leq a_{i, k'} + b_{k', j} \}.
        \]

        Thus we get that $a_{i, k} + b_{k, j} \leq a_{i, \tilde{k}} + b_{\tilde{k}, j}$, because $(i, j) \in X_k$.
    }

    \uncover<3->{
        Applying the definition of $X_{\tilde{k}}$, we then also get $a_{i, \tilde{k}} + b_{\tilde{k}, j} \leq a_{i, k} + b_{k, j}$.
    }

    \uncover<4->{
        This means that $a_{i, \tilde{k}} + b_{\tilde{k}, j} = a_{i, k} + b_{k, j}$.

        To break this tie, we can w.l.o.g.\ assume $k < \tilde{k}$, which would result in $(i, j) \not\in X_{\tilde{k}}$.
    }

    \uncover<5>{
        $\implies$ Every index pair $(i, j)$ can only be included in at most one $X_k$.
    }
\end{frame}

\begin{frame}{Proof of Lemma~\ref{lem:sub_mat_mul}}
    Now assume that there would exist an index pair $(i, j)$ that is contained in none of the $X_k$.
    (Recall $X_k = \{ (i, j) \,|\, \forall k' = 1, \dots, d: a_{i, k} + b_{k, j} \leq a_{i, k'} + b_{k', j} \}$.)
    
    \uncover<2->{
        This is equivalent to the condition that
        \[
            \forall k = 1, \dots, d: \exists k' = 1, \dots, d: a_{i, k} + b_{k, j} > a_{i, k'} + b_{k', j}.
        \]
    }

    \uncover<3->{
        Because of the finite set we can choose $\hat{k} := \argmin\limits_{l = 1, \dots, d} a_{i, l} + b_{l, j}$.
    }

    \uncover<4->{
        Hence, choosing $k = \hat{k}$ yields $\forall k' = 1, \dots, d: a_{i, k} + b_{k, j} = a_{i, \hat{k}} + b_{\hat{k}, j} \leq a_{i, k'} + b_{k', j}$. \Lightning{}
    }
    
    \uncover<5>{
        $\implies$ Every index pair $(i, j)$ has to be included in one $X_k$.
    }
\end{frame}

\begin{frame}{Proof of Lemma~\ref{lem:sub_mat_mul}}
    \begin{alertblock}{Recall}
        Every index pair $(i, j)$ can only be included in at most one $X_k$.

        Every index pair $(i, j)$ has to be included in one $X_k$.
    \end{alertblock}
    
    $\implies$ Over all $X_k, k = 1, \dots, d$, every index pair $(i, j)$ is encountered exactly once.

    \uncover<2>{
        $\implies \begin{aligned}\sum\limits_{k = 1}^d \abs{X_k}\end{aligned} = n^2$

        $ \implies \mathcal{O}\left( \sum\limits_{k = 1}^d \left( c_\varepsilon^d n^{1 + \varepsilon} + \abs{X_k} \right) \right) = \mathcal{O}\left( d c_\varepsilon^d n^{1 + \varepsilon} + n^2 \right)$ \qed{}
    }
\end{frame}

\begin{frame}{Proof of Theorem~\ref{thm:mat_mul}\footnote[1]{\cite[Theorem~3.2]{Chan2007}}}
    \setcounter{theorem}{2}
    \begin{theorem}
        Given any two matrices $A, B \in \mathbb{R}^{n \times n}$ we can compute their min-plus (distance) product in a time of $\mathcal{O}\left( n^3 / \log(n) \right)$.
    \end{theorem}
\end{frame}

\begin{frame}{Proof of Theorem~\ref{thm:mat_mul}}
    We recall the idea of splitting matrices to multiply them:

    \begin{columns}
        \begin{column}{.2\linewidth}
            \begin{figure}
                \begin{tikzpicture}
                    \matrix[hsupermatrix]{
                        \node[vsubmatrix] (a1) {A_1}; \& \node[vsubmatrix] (a2) {A_2}; \& \cdots \& \node[vsubmatrix] (ad) {A_d}; \\
                    };
                \end{tikzpicture}
            \end{figure}
        \end{column}
        \begin{column}{.2\linewidth}
            \begin{figure}
                \begin{tikzpicture}
                    \matrix[vsupermatrix]{
                        \node[hsubmatrix] (b1) {B_1}; \\ \node[hsubmatrix] (b2) {B_2}; \\ \vdots \\ \node[hsubmatrix] (bd) {B_d}; \\
                    };
                \end{tikzpicture}
            \end{figure}
        \end{column}
    \end{columns}

    $\implies$ Strassen is not applicable here --- we need to make use of the relation between matrix multiplication and matrix closure.
\end{frame}

\begin{frame}{Proof of Theorem~\ref{thm:mat_mul}}
    We split our matrices $A$ and $B$ into $\begin{aligned}\frac{n}{d}\end{aligned}$ blocks, that is $\begin{aligned}\forall l = 1, \dots, \frac{n}{d}: A_l \in \mathbb{R}^{n \times d}, B_l \in \mathbb{R}^{d \times n}\end{aligned}$ for some fixed $d$.
    (If necessary we round $\begin{aligned}\frac{n}{d}\end{aligned}$ and adjust the number of blocks accordingly.)

    \uncover<2->{
        We then compute the distance products $A_i \otimes B_i$ for all $\begin{aligned}i = 1, \dots, \frac{n}{d}\end{aligned}$, and set the product to be defined by the element-wise minimum, i.e.\ $\begin{aligned}[t]c_{i, j} := \min\limits_{l = 1, \dots, \frac{n}{d}} {\left( A_l \otimes B_l \right)}_{i, j}\end{aligned}$, where $i, j = 1, \dots, n$.
    }

    \uncover<3->{
        By Lemma~\ref{lem:sub_mat_mul}, this procedure requires $\begin{aligned}\mathcal{O}\left( \frac{n}{d} \left( d c_\varepsilon^d n^{1 + \varepsilon} + n^2 \right) \right) = \mathcal{O} \left( c_\varepsilon^d n^{2 + \varepsilon} + \frac{n^3}{d} \right)\end{aligned}$ time.
    }

    \uncover<4->{
        It now only remains to choose the constant $d$.
    }
\end{frame}

\begin{frame}{Proof of Theorem~\ref{thm:mat_mul}}
    We want to assure $\begin{aligned}\frac{n^3}{d} > c_\varepsilon^d n^{2 + \varepsilon}\end{aligned}$.

    \uncover<2->{
        The choice of $d = \tilde{c} \log(n)$ for $\tilde{c}$ sufficiently small and depending on $\varepsilon$ is useful because
        \begin{itemize}
            \item $C^{D \log(n)} \sim n^D$ grows polynomially, whereas e.\ g.
            \item<3-> $C^n$ grows exponentially.
        \end{itemize}
    }

    \uncover<4->{
        We then get
        $\begin{aligned}
            \mathcal{O}\left( c_\varepsilon^d n^{2 + \varepsilon} + \frac{n^3}{d} \right) = \mathcal{O}\left( \frac{n^3}{\log(n)} \right)
        \end{aligned}$. \qed{}
    }

    \uncover<5>{
        One possible choice is $\varepsilon \approx 0.38, c_\varepsilon \approx 4.32, d \approx 0.42 \log(n)$\footnotemark[1].
    }
\end{frame}

\begin{frame}{Proof of Corollary~\ref{cor:apsp_subcubic}\footnote[1]{\cite[Corollary~3.3]{Chan2007}}}
    \setcounter{theorem}{3}
    \begin{corollary}
        We can solve the all pairs shortest paths problem for a graph $G = (V, E)$ with $\abs{V} = n$ nodes in $\mathcal{O}\left( n^3 / \log(n) \right)$ time.
    \end{corollary}
\end{frame}

\begin{frame}{Proof of Theorem~\ref{cor:apsp_subcubic}}
    We consider $A$ and $B$ to be the matrices defined by $\begin{aligned}d(i, j) := \begin{cases}
        w(e), &\exists e \in E: e = (i, j) \\
        \infty, &else
    \end{cases}\end{aligned}$.

    The corollary then follows by applying Theorem~\ref{thm:mat_mul}. \qed{}
\end{frame}