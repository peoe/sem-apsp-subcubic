\chapter{Problem Setting}

\section{Finding Shortest Paths}

Throughout this manuscript we assume $G = (V, E, \omega)$ to be a directed weighted graph.
In particular we make use of the notations of
\begin{itemize}
    \item a set of \emph{vertices} $V \subseteq \bb{Z}$,
    \item a set of \emph{edges} $E \subseteq \power{V}$, and
    \item a \emph{weight mapping} $\omega \in E' = \{ \varphi: E \rightarrow \bb{R} \mid \varphi \text{ linear} \}$.
\end{itemize}
In combining multiple edges we define a path $p \in \fk{P}(G)$ to be a tuple of edges $(e_1, e_2, \dots, e_n)$ such that each subsequent pair of edges within the path are connected within the graph.
This can be expressed through the condition $\forall i = 1, \dots, n - 1: e_i^{(2)} = e_{i + 1}^{(1)}$, where $e^{(1)}$ denotes the starting vertex and $e^{(2)}$ signifies the end vertex of the edge $e$.
By identifying each edge with its two constituent vertices and keeping track of the number of total elements in the tuple we can equivalently write $p = (e_1, e_2, \dots, e_n) = (v_1, \dots, v_{n + 1})$, where $e_i = (v_i, v_{i + 1})$.
For ease of notation we further denote the subset of all paths from a vertex $u \in V$ to a vertex $v \in V$ by
\[
    \fk{P}(G; u, v) \coloneqq \ISET{p \in \fk{P}(G)}{$p$ starts in $u$ and ends in $v$} \subseteq \fk{P}(G).
\]
For vertices $u, v, w \in V$ and related paths $p_1 = {(u_i)}_{i = 1}^n \in \fk{P}(G; u, v), p_2 = {(v_j)}_{j = 1}^m \in \fk{P}(G; v, w)$ we further define the \emph{concatenation}
\[
    p_1 \oplus p_2 \coloneqq (u_1, u_2, \dots, u_n = v_1, v_2, \dots, v_m).
\]

In applications of graph theory one often wants to find paths in a graph that satisfy certain properties.
Our quantity of interest will be the \emph{weight} any path $p \in \fk{P}(G)$ on a weighted graph $G$ defined by
\[
    \omega(p) \coloneqq \omega\left( e_1 + e_2 + \cdots + e_n \right) = \sum\limits_{i = 1}^n \omega(e_i),
\]
where we applied the linearity of $\omega$ in the equality.
In particular, we will try to find those paths which obtain the minimum weight possible between two vertices.
These so-called \emph{shortest paths} are defined as
\[
    \SP_G(u, v) \coloneqq p^* = \argmin\limits_{p \in \fk{P}(G; u, v)} \omega(p).
\]

In general, the existence and uniqueness of a shortest path cannot be guaranteed.
One example where the existence can never satisfied is a graph with two connected components $V_1, V_2$.
Taking any nodes $u \in V_1, v \in V_2$, one can never find a path from $u$ to $v$, and vice versa.
Such unconnected graphs can be dealt with by choosing an appropriate initialization in the shortest path finding algorithms.
This, in practice, this never causes any problems.
Much more problematic is the existence of \emph{negative weight cycles}, that is a path $c = (v_1, v_2, \dots, v_n), v_1 = v_n = a, a \in V$, with weight $\omega(c) < 0$.
Suppose that for two vertices $u, v \in V$ the vertex $a$ from the negative weight cycle $c$ were reachable on a path $p = (u_1, u_2, \dots, u_{j - 1}, a, u_{j + 1}, \dots, u_n), u_1 = u, u_n = v$, for some $j \in \sset{1, \dots, n}$.
Decomposing $p = (u_1, u_2, \dots, u_{j - 1}, a) \oplus (a, u_{j + 1}, \dots, u_n) \eqqcolon p_1 \oplus p_2$, we can now compute
\[
    \omega(p) = \omega(p_1) + \omega(p_2) > \omega(p_1) + \underbrace{\omega(c)}_{< 0} + \omega(p_2) = \omega(p_1 \oplus c \oplus p_2) > \omega(p_1) + \sum\limits_{i = 1}^\infty \omega(c) + \omega(p_2) = -\infty.
\]
Hence, by repeatedly traversing the negative weight cycle $c$ after reaching vertex $a$ we can decrease the total weight of the path from $u$ to $v$ to an arbitrarily small number.
By this construction, there can be no shortest path from $u$ to $v$ although there clearly exists a path $p \in \fk{P}(G; u, v)$.

\section{Single Source Shortest Paths Problems}

\itodo{sssp}

\section{All Pairs Shortest Paths Problems}

\itodo{apsp}
